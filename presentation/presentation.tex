% document
\documentclass[xcolor=dvipsnames, 10pt]{beamer}

\usepackage[T1]{fontenc}
\usepackage{ragged2e}
\usepackage{graphicx}
\usepackage{hyperref}
\usepackage{csquotes}
\usepackage{caption}
\usepackage{subcaption}
\usepackage{booktabs}
\usepackage[skip=5pt]{parskip}
\usepackage[style = apa]{biblatex}

\addbibresource{presentation.bib}

\usetheme{Darmstadt}
\usecolortheme{lily}

\setbeamertemplate{bibliography item}{\insertbiblabel}	
\setbeamertemplate{navigation symbols}{}

\title[UC3M]{Voting abroad: analysis of 2024 Russian presidential election in immigrant communities}
\subtitle{\vspace{1em}\textit{Presentation for AQMSS II}}
\author[AQMSS II]{Stepan Polikanov and Vera Okisheva}
\date{}
\titlegraphic{
    \includegraphics[width=6cm]{logo_uc3m.png}
}

\begin{document}

\begin{frame}
\titlepage
\end{frame}

\section{Introduction}

\begin{frame}{Why should we care?}
\begin{itemize}
	\item The right to vote abroad is a recent, non-universal and heterogeneous phenomenon (Collyer, 2013), still debated normatively (Bauböck, 2006)
	\item Voters abroad are frequently misaligned with voters at home (Vintila et al, 2023; Battiston and Luconi, 2020; Szulecki et al, 2023)
	\item Political participation in home-country affairs matters for identity:
	\begin{itemize}
		\item Migrant voter turnout is not determined by poor integration in the host society (Gherghina and Basarabă, 2024);
		\item Studies emphasize homesickness and sense of belonging as motivators (Boccagni, 2011)
	\end{itemize}
	\item Very few studies explicitly linking voting abroad to political out-migration or voting in protest
	\begin{itemize}
		\item Autocracies generally extend voting rights to immigrants when they support the incumbent and restrict when they don't (Umpierrez de Reguero et al, 2021; Iams Wellman, 2020)
	\end{itemize}
\end{itemize}
\end{frame}

	\begin{frame}{Broadening the scope}
	Russian data comes in handy for multiple reasons:
	\begin{itemize}
		\item Broad international coverage
		\item Unique multi-country exit poll with good by-station coverage
		\item Comparable data from 2018 and comparable aggregated exit poll data for 2021
	\end{itemize}
	\vspace{1em}
	These enable us to estimate:
	\begin{itemize}
		\item Rough effects of war migration on the change in immigrant political preferences
		\item Selection processes in countries with and without exit polls
		\item Socio-economic determinants of external voting
		\item Destination countries' features' effect on voting
	\end{itemize}
	\end{frame}

	\section{Empirical strategy}

	\begin{frame}{Data}
	\begin{itemize}
		\item Central Election Committee results of the election by voting station:
		\begin{itemize}
			\item 287 voting stations across 145 countries and 261 cities
		\end{itemize}
		\item Exit poll
		\begin{itemize}
			\item 65 voting stations (= cities) across 44 countries
		\end{itemize}
		\item Exit poll raw data
		\begin{itemize}
			\item 69261 respondents in total
			\begin{itemize}
				\item Sex, age, time out of Russia, time to reach voting station, trust in the result and vote choice
			\end{itemize}	
		\end{itemize}
		\item Other data sources for country-level characteristics
	\end{itemize}
	\end{frame}

	\begin{frame}{What we want to test}
	\begin{itemize}
		\item Individual-level predictors, main interest - time of migration (before and after Crimea, after 2022)
		\begin{itemize}
			\item Multinomial models (+ data imputation) and nested logit models
		\end{itemize}
		\item Country-level predictors of vote shares
		\begin{itemize}
			\item Multi-level models with country-level predictors and aggregated voting station/country-level models
			\item Military bases, offshore status, "unfriendly nations", cultural and physical (geoplolitical) distance to Russia
		\end{itemize}
		\item Differences between 2018 and 2024 presidential election results abroad 
		\begin{itemize}
			\item Cumulative effect of war, politization and migration
			\item Models for strictly exit poll sample and the whole sample
		\end{itemize}
		\item Differences between exit poll and official results
		\begin{itemize}
			\item Ecological analysis for falsification possibilities
		\end{itemize}
	\end{itemize}
	\end{frame}

\end{document}